
\section{Tips}

  \begin{itemize}

\item \textbf{DO NOT GIVE OR ASK YOUR CODE TO/FROM ANYBODY.} Not even
  an old version. Not even to your best friend. Not even from students
  of previous years. We will use plagiarism detectors to compare
  pairwise all submissions and also with submissions from previous
  editions.  However, you can share the compiled \texttt{.o} files.
  
  Any detected plagiarism will result in an {\bf overall grade of 0}
  in the course (not only in the Game) of all involved students.
  Additional disciplinary measures might also be taken. If student A
  and B are involved, measures will be applied to both of them,
  independently of who created the original code. No exceptions will
  be made under any circumstances.

\item Before competing with your classmates, focus on qualifying and
defeating the ''Dummy'' player.

\item Read only the headers of the classes in the provided source
  code. Do not worry about the private parts nor the implementation.

\item Start with simple strategies, easy to code and debug, since this
is exactly what you will need at the beginning.

\item Define basic auxiliary methods, and make sure they work
  properly.

\item Try to keep your code clean. Then it will be easier to change it
and to add new strategies.

\item As usual, compile and test your code often. It is \emph{much}
easier to trace a bug when you only have changed few lines of code.

\item Use \texttt{cerr}s to output debug information and add
  \texttt{assert}s to make sure the code is doing what it should do.
  Remember to remove (or comment out) the \texttt{cerr}s before
  uploading your code to Jutge.org. Otherwise, your submission will be
  killed.


\item When debugging a player, remove the \texttt{cerr}s you may have in
the other players' code, to make sure you only see the messages you
want.

\item By using commands like \texttt{grep} in Linux you can filter the
  output that \texttt{Game} produces.
  
\item Switch on the \texttt{DEBUG} option in the Makefile, which will
allow you to get useful backtraces when your program crashes. There is
also a \texttt{PROFILE} option you can use for code optimization.

\item If using \texttt{cerr} is not enough to debug your code, learn
  how to use \texttt{valgrind}, \texttt{gdb}, \texttt{ddd} or any
  other debugging tool. 

\item You can analyze the files that the program \texttt{Game}
produces as output, which describe how the board evolves after each
round.

\item Keep a copy of the old versions of your player. When a new
version is ready, make it fight against the previous ones to measure
the improvement.

\item Make sure your program is fast enough: the CPU time you are
allowed to use is rather short.

\item Try to figure out the strategies of your competitors by
watching matches. This way you can try to defend against them or
even improve them in your own player.

\item Do not wait till the last minute to submit your player. When
there are lots of submissions at the same time, it will take longer for
the server to run the matches, and it might be too late!

\item You can submit new versions of your program at any time.

\item And again: Keep your code simple, build often, test often. Or
you will regret.

\end{itemize}


