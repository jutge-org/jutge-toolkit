
\section{Programming the game}

The first thing you should do is downloading the source code. It
includes a C++ program that runs the games and an HTML viewer to watch
them in a reasonable animated format. Also, a ``Null'' player and a
``Demo'' player are provided to make it easier to start coding your own
player.

\subsection{Running your first game}

Here, we will explain how to run the game under Linux, but it should
work \mbox{under} Windows, Mac, FreeBSD, OpenSolaris, \dots You only need a
recent \texttt{g++} version, \texttt{make} installed in your system, plus a
modern browser like Firefox or Chrome.


\begin{enumerate}

\item Open a console and \texttt{cd} to the directory where you
extracted the source code.

\item If, for example, you are using a 64-bit Linux version, run:

  \texttt{cp AIDummy.o.Linux64 AIDummy.o}
  
  \texttt{cp Board.o.Linux64 Board.o}

  If you use any other architecture, choose the right objects you will
  find in the directory.

\item Run

\texttt{make all}

to build the game and all the players. Note that \texttt{Makefile}
identifies as a player any file matching \texttt{AI*.cc}.

\item This creates an executable file called
\texttt{Game}. This executable allows you to run a game using a command
like:

\texttt{./Game Demo Demo Demo Demo -s 30 -i default.cnf -o default.res}

This starts a match, with random seed 30, of four instances of the
player Demo, in the board defined in \texttt{default.cnf}. The output of
this match is redirected to \texttt{default.res}.

\item To watch a game, open the viewer file \texttt{viewer.html} with
  your browser, for example by running \texttt{firefox viewer.html},
  and load the file \texttt{default.res}.

\end{enumerate}


Use

\texttt{./Game --help}

to see the list of parameters that you can use. Particularly useful is

\texttt{./Game --list}

to show all the recognized player names.

\medskip

If needed, remember that you can run

\texttt{make clean}

to delete the executable and object files and start over the build.


\begin{center}
\includegraphics[width=12cm]{screenshot.png}
\end{center}


\subsection{Adding your player}

To create a new player with, say, name \texttt{Rick},
copy \texttt{AINull.cc} (an empty player that is provided as a template)
to a new file \texttt{AIRick.cc}.
Then, edit the new file and change the

\begin{center}
@#define PLAYER_NAME Null@
\end{center}

line to

\begin{center}
@#define PLAYER_NAME Rick@
\end{center}

The name that you choose for your player must be unique, non-offensive
and at most 12 characters long. This name will be shown in the website
and during the matches.

\medskip

Afterwards, you can start implementing the virtual method @play()@,
inherited from the base class @Player@. This method, which will be
called every round, must decide the orders to give to your units.

\medskip

You can define auxiliary type definitions, variables and methods inside
your player class, but the entry point of your code will always be the
@play()@ method.

\medskip

From your player class you can also call functions that you will find in the following files:

\begin{itemize}
\item \texttt{State.hh}: consulting the state of the game.
\item \texttt{Action.hh}: giving orders to your units.
\item \texttt{Structs.hh}: useful data structures.
\item \texttt{Settings.hh}: consulting the initial parameters of the game.
\item \texttt{Player.hh}: method @me()@.
  \item \texttt{Random.hh}: random number generation.
\end{itemize}

You will find a summary of this information in the file \texttt{api.pdf}.
You can also examine the code of the ``Demo'' player in \texttt{AIDemo.cc}
as an example of how to use these functions.

\medskip

Note that you must not edit the @factory()@ method of your player
class, nor the last line that adds your player to the list of available
players.



\subsection{Restrictions when submitting your player}

When you think that your player is strong enough to enter the
competition, you can submit it to the Jutge. Since it will run in a
secure environment to prevent cheating, some restrictions apply to your
code:

\begin{itemize}

\item All your source code must be in a single file (like AIRick.cc).

\item You cannot use global variables (instead, use attributes in your
class).

\item You are only allowed to use standard libraries like
\texttt{iostream}, \texttt {vector}, \texttt {map}, \texttt {set},
\texttt {queue}, \texttt {algorithm}, \texttt {cmath}, \dots In many
cases, you don't even need to include the corresponding library.

\item You cannot open files nor do any other system calls (threads,
forks, \dots).

\item Your CPU time and memory usage will be limited, while they are
not in your local environment when executing with \texttt{./Game}.

\item Your program should not write to @cout@ nor read from @cin@. You
can write debug information to @cerr@, but remember that doing so in
the code that you upload can waste part of your limited CPU time.

\item Any submission to the Jutge must be an honest attempt to play the
game. Any try to cheat in any way will be severely penalized.

\item Once you have submitted a player to Jutge that has defeated the Dummy
  player, you can send more submissions but you will have to change the
  player name. That is, once a player has defeated Dummy, his name is
  blocked and cannot be reused.

\end{itemize}

