
\section{Game Rules}

Due to yet unknown reasons, the Earth is going through a zombie
apocalypse. The few human survivors fight to stay alive and to gain
control of the scarce existing resources.

This is a game for four players, identified with numbers from 0 to
3. Each player has control over a clan of alive units. But there are
two additional types of units in the game: dead units and zombies.

The game lasts 200 rounds, numbered from 1 to 200. Each unit
can move at most once per round. Dead units, as expected, cannot move.
During the rounds, the clans accumulate points and the winner
of the game is the clan with the largest amount of points after
round 200.

The board game has dimensions $60 \times 60$. Units cannot move
outside it.  A position in the board is given by a pair of integers
$(r,c)$ where $0 \leq r < 60$ and $0\leq c < 60$. The top-left
position is $(0,0)$, whereas the bottom-right position is
$(59,59)$. Hence, $r$ (for row) refers to the vertical axis and $c$
(for column) refers to the horizontal axis. Each cell in the board is
either part of a street or is full of waste. Units cannot move on
waste and must necessarily move through streets.

Clans start the game with a certain amount of strength points.  The
{\em strength of a clan} is defined as $\left\lfloor{{\texttt{strength
      points}}\over{\texttt{alive units}}}\right\rfloor$ and is key
for determining the winner of the fights that will occur during the
game.  After each round, the strength points of a clan will be
decremented in an amount equal to the number of alive units of that
clan. However, strength points are never negative. In order to
increment strength points, alive units can collect food. It is easy to
see that a clan with a lot of alive units needs to collect a lot of
food in order to have a good amount of strength.  Thus, some units
decide to abandon their clan: at each round, with a probability of
$20\%$, a unit from the clan with the most number of alive units
becomes part of the clan with the least amount of alive units. If
there are several clans with these properties, one of them is chosen
randomly.

\bigskip
    {\bf Movements of alive units.} An alive unit moves in the board following these rules:
    
\begin{itemize}
\item It can only move to adjacent cells horizontally and vertically, never in diagonal.
  
\item If it tries to move to a cell occupied by waste, by a unit of
  the same clan, o by a dead unit, the movement will be ignored.

\item If it moves to a cell with food, the strength points of its clan will be incremented,
  the unit will occupy the cell, the food item will disappear and the clan will possess this cell.
  At the end of the round, a food item will reappear in another position. Note that we will never
  find a cell occupied by food and by a unit.

\item It it tries to move to an empty cell (without food or unit),
  the movement will be done and the clan will possess this cell.

\item If it moves to a cell occupied by a zombie, the zombie will die but the unit will not move. As a result,
  the clan will receive a certain number of points and, at the end of the round, an alive unit
  of this clan will reappear in another position, and the zombie will disappear.

\item If it moves to a cell occupied by an alive unit of another clan, a fight will start. The
  unit losing the fight will become a dead unit, and the process of zombie conversion will start: after a certain
  number of rounds, it will be a zombie. If this unit had already started a zombie conversion process for being
  bitten by a zombie, the process will restart. The clan of the unit winning the fight will receive
  a certain number of points, but the unit will not move. The winner of the fight is determined as follows:

  With a probability of $30\%$, the unit starting the fight will
  surprise the other unit and hence will win the fight
  immediately. Otherwise, if the strengths of the clans involved in
  the fight are $N$ and $M$, respectively, the first unit will win
  with probability $N/(N+M)$ and the second unit with probability
  $M/(N+M)$. If $N = M = 0$, the units will have the same probability
  of winning.
\end{itemize}

{\bf Movements of zombies.} Zombies are not part of any clan and hence
are not controlled by any player.  Zombies will always move to the
closest alive unit, considering that they cannot move through
waste. If there are multiple units at the same distance, one of them
will be randomly chosen. A zombie will always move following these
rules:

\bigskip
\begin{itemize}
  
\item It can move to adjacent cells horizontally, vertically and also in diagonal.
    
\item It will never move to a cell occupied by waste, a dead unit or a zombie.
    
\item If it moves to a cell occupied by food, the food will
  disappear. At the end of the round, a food item will reappear in a
  random position. If a clan was possessing this cell, it will stop
  doing so.

\item If it moves to an empty cell (with no food or unit)
  owned by a clan, this clan will stop possessing this cell.

\item If it tries to move to a cell with an alive unit on it, the movement will not be done.
  However, it will bite the unit and it will start the process of becoming a zombie, that will finish
  after a certain number of rounds. If that process had already started due to a previous bite,
  the process will not restart, but rather continue. During the conversion, the unit will behave
  as an alive one.
\end{itemize}

As a result of the previous rules, the total number of units is constant during the game.

\bigskip
    {\bf Object regeneration.}
    Every time a food unit or an alive unit needs to be regenerated, it will always appear in an
    empty cell $C$ with no unit or food in the surrounding cells (the ones with an x in the table):

\medskip
\begin{center}
  \begin{tabular}{|c|c|c|c|c|}\hline
    x & x & x & x & x \\\hline
    x & x & x & x & x \\\hline
    x & x & C & x & x \\\hline
    x & x & x & x & x \\\hline
    x & x & x & x & x \\\hline
  \end{tabular}
\end{center}
\medskip

If there is no safe cell in this sense, the object will reappear in an empty cell,
with no unit or food in it.

It is important to remark that units have an identifier that never changes, not even
after they are regenerated. For example, if a unit with a certain initial identifier becomes
a zombie, it will still have the same identifier. If, later on, the zombie is killed and reappears
as an alive unit of another clan, the identifier will still be the same.

\bigskip
    {\bf Score computation.} The score of a clan in a given round is given by two components.
    On the one hand, for each zombie that the clan has killed so far, 10 points will be awarded.
    For each killed alive unit, 50 points will be given.

    On the other hand, for each cell owned by the clan {\bf in this
      round} 1 point is given. The score of a clan is the sum of these
    two components. Note that the score can be decremented if a clan
    loses the possession of a cell. The constants 10, 50 and 1, as
    well as other constants that define the initial settings of the
    game, are defined in the input file \texttt{default.cnf}.  All
    games will be played with the values given in this file.
    
\bigskip
{\bf Execution of orders.}
In each round, more than one order can be given to the same unit,
although only the first such order (if any) will be selected. Any player
that tries to give more than 1000 orders during the same round will be aborted.

Every round, the selected movements of the four players will be executed using a random order,
but respecting the relative order of the units of the same clan.
As a consequence of the previous rule, consider giving the orders to your units at every
round from most urgent to least urgent.

Take into account that the every movement is applied to the board resulting
of the previous movements. For example, consider the board

\medskip
\begin{center}
  \begin{tabular}{|c|c|c|c|c|}\hline
    x & x & x & x \\\hline
    x & F & U & x \\\hline
    x & V & x & x \\\hline
    x & x & x & x \\\hline
  \end{tabular}
\end{center}
\medskip

where F represents food and U, V two alive units of different clans.
Let us assume that the player controlling U decides that it should go
left, and the player controlling V decides to go up. If the V movement
is executed first, then U has no food left, because V has already
taken it. Moreover, the subsequent movement of U is an attack to V. If
U dies in this fight, in the visualization of the game we will see a
transition from the previous matrix to a situation where U has
disappeared. Obviously, there could be no attack between U and V in
the original configuration because units cannot move diagonally, but
the execution order of the movements has made it possible. Take this
into account when you do not understand certain situations in the
visualization of games.

After the execution of all movements from the players, the movements
of zombies are executed.  After that, the following processes are
executed in this order: units that have finished their conversion
process become zombies, killed zombies will reappear as alive units, a
unit from the clan with the largest number of alive units might move
to the clan with the least number of alive units, food items will be
regenerated and scores will be updated.


