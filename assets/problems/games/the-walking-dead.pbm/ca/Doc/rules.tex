
\section{Regles del Joc}

Per motius que encara ningú coneix amb certesa, l'apocalipsi zombi
s'ha apoderat de la Terra. Els pocs humans que hi queden intenten
sobreviure i apoderar-se dels escassos recursos encara existents.

Es tracta d'un joc per a quatre jugadors, identificats amb números de
0 a 3. Cada jugador té el control d'un clan d'unitats vives.  Però al
joc coexisteixen dos tipus més d'unitats: unitats mortes i zombis.

El joc té una durada de 200 rondes, numerades de l'1 al 200. Cada
unitat pot moure's com a màxim una vegada per ronda. Les unitats
mortes, com és d'esperar, no es poden moure. Durant aquestes rondes
els clans aniran acumulant punts i guanyarà la partida qui tingui més
punts en acabar la ronda 200.

El tauler del joc té dimensions $60 \times 60$.  Les unitats no es
poden moure en cap cas fora d'ell. Una posició del tauler ve
determinada per un parell d'enters $(f,c)$ on $0 \leq f < 60$ i $0\leq
c < 60$. La posició de més a dalt i a l'esquerra és la $(0,0)$, mentre
que la de més a baix i a la dreta és $(59,59)$. Per tant, la primera
coordenada ($f$ de fila) és la que indica la posició en l'eix vertical
i la segona ($c$ de columna) en l'eix horitzontal.  Cada cel·la del
tauler o bé forma part d'un carrer o bé està plena de deixalla. Les
unitats no poden trepitjar la deixalla i s'han de moure necessàriament
pels carrers.

Els clans comencen la partida amb una certa quantitat de punts de
força.  La {\em força d'un clan} es defineix com
$\left\lfloor{{\texttt{punts força}}\over{\texttt{unitats
      vives}}}\right\rfloor$ i serà clau per determinar el guanyador
de les lluites que hi haurà durant la partida. En acabar cada ronda,
els punts de força d'una clan es decrementaran en una quantitat igual
al nombre d'unitats vives d'aquest clan. No obstant, mai esdevindrà
una quantitata negativa. Per tal d'incrementar els punts de força, les
unitats vives poden agafar menjar que trobaran al tauler. És fàcil
veure que un clan amb moltes unitats vives necessita recollir molt
menjar per a poder mantenir una força considerable. És per això que
algunes unitats decideixen canviar de clan: cada ronda, amb un $20\%$
de probabilitat, una unitat del clan amb més unitats vives passa a
formar part del clan amb menys unitats vives. En cas d'haver-hi
diferents clans amb aquestes propietats, se'n selecciona un d'ells de
forma aleatòria.

\bigskip
{\bf Moviments de les unitats vives.} Una unitat viva d'un clan es
pot moure pel tauler de la manera següent:

\begin{itemize}
  \item Només pot accedir a les cel·les adjacents en horitzontal i
    vertical, mai en diagonal.

  \item Si es mou cap a una cel·la ocupada per deixalla, per una
    unitat del seu mateix clan o per una unitat morta, el moviment
    s'ignorarà.

  \item Si es mou cap a una cel·la ocupada per menjar, els punts de
    força del seu clan s'incrementaran, la unitat ocuparà la cel·la
    tot fent desaparèixer el menjar i el seu clan posseirà aquesta
    cel·la. En acabar la ronda, una unitat de menjar reapareixerà en
    una altra posició. Notem que no trobarem mai una cel·la ocupada
    per menjar i per una unitat.

  \item Si es mou cap a una cel·la buida (sense menjar ni cap unitat),
    el moviment s'efectuarà i el seu clan passarà a posseir aquesta
    cel·la.

  \item Si es mou cap a una cel·la ocupada per un zombi, el zombi
    morirà però la unitat no es mourà. Com a resultat, el clan rebrà
    un cert nombre de punts i, en acabar la ronda, una unitat viva
    d'aquest clan reapareixerà en una altra posició reemplaçant al
    zombi, que desapareixerà.

  \item Si es mou cap a una cel·la ocupada per una unitat viva d'un
    altre clan, es produirà una lluita. La unitat que perdi la lluita
    passarà a ser una unitat morta, i s'inicia el seu procés de
    conversió: al cap d'un cert nombre de rondes, passarà a ser un
    zombi. Si ja havia començat el procés de conversió per haver estat
    mossegat per un zombi, el procés torna a començar. El clan de la
    unitat que guanyi l'atac rebrà un cert nombre de
    punts, però no es mourà de posició. El guanyador de la lluita
    es determina de la manera següent:

    Amb un $30\%$ de probabilitat, la unitat que ha desencadenat
    l'atac agafarà l'altra per sorpresa i serà la guanyadora sense
    haver de lluitar. En cas contrari, si les forces dels dos clans
    involucrats en la lluita són $N$ i $M$, respectivament, el primer
    guanyarà amb probabilitat $N/(N+M)$ i el segon amb probabilitat
    $M/(N+M)$. En cas que $N = M = 0$, les dues unitats tindran la
    mateixa probabilitat de guanyar.   
\end{itemize}

{\bf Moviments dels zombis.} Els zombis no formen part de cap clan i
per tant no podran ser controlats per cap jugador. Els zombis sempre
es mouran cap a la unitat viva més propera a ells, tenint en compte la
presència d'obstacles. En cas d'haver-n'hi diverses a la mínima
distància, n'escolliran una a l'atzar. Els moviments d'un zombi es
regiran per les següent regles:

\bigskip
\begin{itemize}
  
\item Pot accedir a les cel·les adjacents en horitzontal i vertical, i
  també en diagonal.
  
\item Mai es mourà cap a una cel·la ocupada per deixalla, per una
  unitat morta o per un zombi.
  
\item Si es mou cap a una cel·la ocupada per menjar, el menjar
  desapareixerà. En acabar la ronda, una unitat de menjar reapareixerà
  en una posició aleatòria. Si un clan posseïa aquesta cel·la, deixarà
  de fer-ho.

\item Si es mou cap a una cel·la buida (sense menjar ni cap unitat)
  posseïda per un clan, aquest perdrà la possessió.

\item Si es vol moure cap a una cel·la ocupada per una unitat viva, el
  moviment no s'efectuarà.  No obstant, mossegarà a la unitat i
  aquesta iniciarà el seu procés de conversió a zombi, que tindrà lloc
  al cap d'un cert nombre de rondes. Si aquest procés ja s'havia
  iniciat per una mossegada prèvia, el procés de conversió no tornarà
  a començar sinó que continuarà el seu curs. Durant la conversió a zombi,
  la unitat es comportarà igual que una unitat viva.

\end{itemize}

Com a resultat de les anterior regles, el nombre d'unitats totals és
constant al llarg de tot el joc.

\bigskip
{\bf Regeneració d'objectes.} Cada vegada que cal regenerar una unitat
de menjar o una unitat viva, aquesta reapareixerà sempre en una
cel·la buida $C$ i tal que no hi ha cap unitat ni menjar en les posicions que
l'envolten (les marcades amb una x a la taula):

\medskip
\begin{center}
  \begin{tabular}{|c|c|c|c|c|}\hline
    x & x & x & x & x \\\hline
    x & x & x & x & x \\\hline
    x & x & C & x & x \\\hline
    x & x & x & x & x \\\hline
    x & x & x & x & x \\\hline
  \end{tabular}
\end{center}
\medskip

Si en el moment de reaparèixer no existeix cap cel·la segura en aquest
sentit, l'objecte reapareixerà en una cel·la buida, que no tingui cap unitat ni menjar.

És important remarcar que les unitats tenen un identificador que mai
canvia, ni tan sols durant el procés de regeneració. És a dir, si una
unitat amb un cert identificador inicial es converteix en zombi,
aquest continuarà amb el mateix identificador. Si posteriorment el
zombi mor i reapareix com una unitat viva d'un altre clan,
l'identificador es mantindrà.

\bigskip
{\bf Càlcul de la puntuació.} La puntuació d'un clan en una ronda ve
determinada per dos components. D'una banda, per cada zombi que el
clan hagi matat fins aquell moment s'obtindran
10 punts.  Per cada unitat que hagi
matat s'obtindran 50 punts.

D'altra banda, per cada cel·la posseïda pel clan {\bf en aquesta
  ronda} s'obtindrà 1 punt.  La puntuació del clan és la suma
d'aquestes quantitats. Així doncs, la puntuació es pot decrementar si
es perd la possessió d'algunes cel·les. Les constants 10, 50 i 1, així
com d'altres que especifiquen els paràmetres inicials del joc, estan
definides a l'arxiu d'entrada \texttt{default.cnf}. Totes les partides
es jugaran amb exactament els valors donats en aquest arxiu.

\bigskip
{\bf Execució d'ordres.}
A cada ronda es pot donar més d’una ordre a la mateixa unitat,
tot i que només se seleccionarà la primera d'elles (si n’hi ha alguna).
Tot programa que intenti donar més de 1000 comandes
durant la mateixa ronda s'avortarà.

Cada ronda, les ordres seleccionades dels quatre jugadors
s'executaran amb ordre aleatori,
però respectant l’ordre relatiu entre les unitats d’un mateix clan.
Com a conseqüència de la norma anterior, considereu la possibilitat de
donar les ordres a les vostres unitats a cada ronda de més urgent a
menys urgent.

Tingueu en compte que s’aplica cada moviment sobre el tauler
que resulta dels moviments anteriors. Per exemple, considereu el tauler

\medskip
\begin{center}
  \begin{tabular}{|c|c|c|c|c|}\hline
    x & x & x & x \\\hline
    x & M & U & x \\\hline
    x & V & x & x \\\hline
    x & x & x & x \\\hline
  \end{tabular}
\end{center}
\medskip

on M representa menjar i  U i V dues unitats vives de diferents clans.
Imaginem que el jugador que controla U ha decidit que aquest vagi cap a l'esquerra,
i el jugador que controla V ha decidit que vagi amunt. Si s'executa
primer el moviment de V, aleshores U s'ha quedat sense menjar,
perquè V ja l'ha consumit i a més, la posterior execució del moviment de U és
un atac cap a V del qual en pot sortir malparat. Si U mor durant aquest atac,
en la visualització de la partida veurem una transició de la matriu anterior
cap a una situació on U ha desaparegut. Òbviament, no hi podia haver un atac entre U i V des
de la configuració inicial, perquè les unitats vives no es poden moure en diagonal, però l'ordre
d'execució de les comandes sí que ho ha fet possible. Tingueu això en compte quan no entengueu
certes situacions durant la visualització de les partides.

Després de l'execució de tots els moviments dels jugadors, es
procedeixen a efectuar els moviments dels zombis.  En acabar, els
següents processos s'executaran en aquest ordre: les unitats que han
acabat el seu procés de conversió passaran a ser zombis, els zombis
morts es convertiran en unitats vives, una unitat del clan amb més
unitats vives potser passarà a formar part del clan amb menys unitats
vives, es regeneraran les unitats de menjar consumides i
s'actualitzaran les puntuacions.

