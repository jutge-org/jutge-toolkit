\section{Com programar un jugador}
El primer que heu de fer és descarregar-vos el codi font. Aquest
inclou un programa C ++ que executa les partides i un visualitzador HTML
per veure-les en un format raonable i animat. A més, us proporcionem un
jugador ``Null'' i un jugador ``Demo'' per facilitar el començament de
la codificació del jugador.

\subsection{Executar la primera partida}
Aquí us explicarem com executar el joc sota Linux, però hauria de funcionar
també sota Windows, Mac, FreeBSD, OpenSolaris, ... Només necessiteu una
versió recent \texttt{g ++}, el \texttt{make} instal·lat al sistema, a més d'un
navegador modern com Firefox o Chrome.

\begin{enumerate}

\item Obriu una consola i feu \texttt{cd} al directori on us heu
  descarregat el codi font.

\item Si, per exemple, teniu una versió de Linux en 64 bits, executeu:

  \texttt{cp AIDummy.o.Linux64 AIDummy.o}
  
  \texttt{cp Board.o.Linux64 Board.o}

  Amb altres arquitectures, cal escollir els objectes adequats que
  trobareu al directori.
  
\item Executeu

  \texttt{make all}

  per compilar el joc i tots els jugadors. Tingueu en compte que el \texttt{Makefile}
identifica com a jugador qualsevol fitxer que coincideixi amb \texttt{AI*.cc}


\item Es crea un fitxer executable anomenat \texttt{Game}. Aquest
  executable us permet executar una partida mitjançant una comanda
  com la següent:
  

\texttt{./Game Demo Demo Demo Demo -s 30 < default.cnf > default.res}

Aquesta comanda comença una partida, amb la llavor aleatòria 30, amb
quatre instàncies del jugador Demo, al tauler definit a
\texttt{default.cnf}. La sortida d'aquesta partida es redirigeix a
\texttt{default.res}.


\item Per veure una partida, obriu el fitxer visualitzador \texttt{viewer.html} amb
  el navegador, per exemple executant des d'un terminal la comanda \texttt{firefox
  viewer.html}, i carregueu el fitxer \texttt{default.res}.

\end{enumerate}


Utilitzeu

\texttt{./Game --help}

per veure la llista de paràmetres que es poden usar. Particularment útil és

\texttt{./Game --list}

per veure tots els noms de jugadors reconeguts.

\medskip

En cas que sigui necessari, recordeu que podeu executar

\texttt{make clean}

per esborrar l'executable i els objectes i començar la compilació de nou.

\begin{center}
\includegraphics[width=12cm]{screenshot.png}
\end{center}


\subsection{Afegir el vostre jugador}

Per crear un jugador nou amb, per exemple, nom \texttt{Rick},
copieu \texttt{AINull.cc} (un jugador buit que proporcionem com a plantilla)
a un fitxer nou \texttt{AIRick.cc}.
A continuació, editeu el fitxer nou i canvieu la línia

\begin{center}
@#define PLAYER_NAME Null@
\end{center}

a

\begin{center}
@#define PLAYER_NAME Rick@
\end{center}

El nom que trieu pel vostre jugador ha de ser únic, no ofensiu
i tenir com a màxim 12 caràcters. Aquest nom es mostrarà al lloc web
i durant les partides.

\medskip
A continuació, podeu començar a implementar el mètode virtual @play()@,
heretat de la classe base @Player@. Aquest mètode, que serà
cridat a cada ronda, ha de determinar les ordres que s'enviaran a les vostres unitats.

Podeu utilitzar definicions de tipus, variables i mètodes a
la vostra classe de jugador, però el punt d’entrada del vostre codi serà sempre el
mètode @play()@.

Des de la vostra classe jugador també podeu cridar funcions que
trobareu especificades als arxius següents:

\begin{itemize}
\item \texttt{State.hh}: accedir a
  l'estat del joc.
\item \texttt{Action.hh}: donar ordres a les vostres unitats.
\item \texttt{Structs.hh}: estructures de dades útils.
\item \texttt{Settings.hh}: accedir als paràmetres del joc.
\item \texttt{Player.hh}: mètode @me()@.
  \item \texttt{Random.hh}: generar nombre aleatoris.
\end{itemize}

Trobareu un resum de tota aquesta informació a l'arxiu \texttt{api.pdf}.
També podeu examinar el codi del
jugador ``Demo'' a \texttt{AIDemo.cc} com a exemple de com usar
aquestes funcions.

Tingueu en compte que no heu d'editar el mètode @factory()@ de la
classe del vostre jugador, ni l'última línia que afegeix el vostre
jugador a la llista de jugadors disponibles.


\subsection{Restriccions en enviar el vostre jugador}

Quan creieu que el vostre jugador és prou fort per entrar a la
competició, podeu enviar-lo al Jutge. Degut a que s'executarà en un entorn segur
per prevenir trampes, algunes restriccions s'apliquen al vostre
codi:

\begin{itemize}

  
\item Tot el vostre codi font ha d’estar en un sol fitxer (com AIRick.cc).

\item No podeu utilitzar variables globals (en el seu lloc, utilitzeu atributs a la vostra
classe).

\item Només teniu permès utilitzar biblioteques estàndard com
\texttt{iostream}, \texttt{vector}, \texttt{map}, \texttt{set},
\texttt{queue}, \texttt{algoritme}, \texttt{cmath}, \dots En molts
casos, ni tan sols cal incloure la biblioteca corresponent.

\item No podeu obrir fitxers ni fer cap altra crida a sistema (threads, forks, ...)
  
\item El vostre temps de CPU i la memòria que utilitzeu seran limitats, mentre que no ho són
al vostre entorn local quan executeu \texttt{./Game}.

\item El vostre programa no ha d'escriure a @cout@ ni llegir de @cin@. Podeu
  escriure informació de depuració a @cerr@, però recordeu que fer això en el
  codi que envieu al Jutge pot fer malgastar innecessàriament temps de CPU.

\item Qualsevol enviament al Jutge ha de ser un intent honest de jugar.
  Qualsevol intent de fer trampes de qualsevol manera serà durament penalitzat.

\item Un cop hagueu enviat un jugador al Jutge que hagi derrotat al
  Dummy, podeu fer més enviaments però haureu de canviar el nom
  del jugador. És a dir, un cop un jugador ha vençut al Dummy, el
  seu nom queda bloquejat i no es pot reutilitzar.

\end{itemize}


%% \section{Com programar un jugador}
%% El primer que heu de fer és descarregar-vos el codi font. Aquest
%% inclou un programa C ++ que executa les partides i un visualitzador HTML
%% per veure-les en un format raonable i animat. A més, us proporcionem un
%% jugador ``Null'' i un jugador ``Demo'' per facilitar el començament de
%% la codificació del jugador.

%% \subsection{Executar la primera partida}
%% Aquí us explicarem com executar el joc sota Linux, però hauria de funcionar
%% també sota Windows, Mac, FreeBSD, OpenSolaris, ... Només necessiteu una
%% versió recent de \texttt{g ++}, el \texttt{make} instal·lat al sistema, a més d'un
%% navegador modern com Firefox o Chrome.

%% \begin{enumerate}

%% \item Obriu una consola i feu \texttt{cd} al directori on us heu
%%   descarregat el codi font.

%% \item Si, per exemple, teniu una versió de Linux en 64 bits, executeu:

%%   \texttt{cp AIDummy.o.Linux64 AIDummy.o}
  
%%   Amb altres arquitectures, cal escollir els objectes adequats que
%%   trobareu al directori.
  
%% \item Executeu

%%   \texttt{make all}

%%   per compilar el joc i tots els jugadors. Tingueu en compte que el \texttt{Makefile}
%% identifica com a jugador qualsevol fitxer que coincideixi amb \texttt{AI*.cc}


%% \item Es crea un fitxer executable anomenat \texttt{Game}. Aquest
%%   executable us permet executar una partida mitjançant una comanda
%%   com la següent:
  

%% \texttt{./Game Demo Demo Demo Demo -s 30 < default.cnf > default.res}

%% Aquesta comanda comença una partida, amb la llavor aleatòria 30, amb
%% quatre instàncies del jugador Demo, al tauler definit a
%% \texttt{default.cnf}. La sortida d'aquesta partida es redirigeix a
%% \texttt{default.res}.


%% \item Per veure una partida, obriu el fitxer visualitzador \texttt{Viewer/viewer.html} amb
%%   un navegador i carregueu el fitxer \texttt{default.res}.

%% \end{enumerate}


%% Utilitzeu

%% \texttt{./Game --help}

%% per veure la llista de paràmetres que es poden usar. Particularment útil és

%% \texttt{./Game --list}

%% per veure tots els noms de jugadors reconeguts.

%% \medskip

%% En cas que sigui necessari, recordeu que podeu executar

%% \texttt{make clean}

%% per esborrar l'executable i els objectes i començar la compilació de nou.

%% %% \begin{center}
%% %% \includegraphics[width=12cm]{screenshot.eps}
%% %% \end{center}

%% \subsection{Arxiu de configuració}

%% Us proporcionem dos exemples d'arxius de configuració. En totes les
%% partides que es juguin al Jutge, incloent les eliminatòries i la
%% final, utilitzarem sempre \texttt{default.cnf}. Aquest arxiu
%% fixa els paràmetres de la Figura~\ref{fig:parameters} al valor per
%% defecte. La posició dels edificis, dels ciutadans, de les armes i dels
%% bonus es decideixen de forma aleatòria.

%% L'arxiu de configuració \texttt{default-fixed.cnf} mostra com es poden
%% canviar els paràmetres i definir el tauler d'entrada. Els caràcters
%% de la graella són '.'(carrer), 'B' (edifici), 'G' (pistola), 'Z'
%% (bazuca), 'M' (diners), 'F' (menjar). Pel que fa als ciutadans, el seu
%% tipus és 'w' (guerrer) o bé 'b' (constructor), i les armes possibles
%% són 'n' (cap arma), 'h' (martell), 'g' (pistola) i 'b' (bazuca).

%% Podeu crear arxius de configuració vosaltres mateixos per diversió o
%% si voleu provar els vostres jugadors en taulers més petits o amb
%% alguna peculiaritat. Els valors dels paràmetres han d'estar dins el
%% rang establert a la Figura~\ref{fig:parameters}. Heu d'assegurar-vos
%% que les forces d'atac i de demolició del bazuca han de ser majors o
%% iguals que les de la pistola, i les de la pistola majors o iguals que
%% les del martell. Una altra restricció és que BARRICADE\_RESISTANCE\_STEP
%% ha de ser menor o igual que BARRICADE\_MAX\_RESISTANCE.  Finalment, si
%% voleu modificar els paràmetres i generar el tauler aleatòriament, heu
%% d'assegurar-vos que el tauler és prou gran per encabir tots els
%% objectes.

%% \begin{figure}
%%   \centering
%%   \begin{tabular}{|l|c|c|} \hline

%%   \textbf{Paràmetre}    & \textbf{Valor per defecte} & \textbf{Rang} \\ \hline\hline

%%   @ NUM_PLAYERS               @  &    4     &   $[4,4]$             \\
%%   @ NUM_DAYS                  @  &    5     &   $[1,\infty)$           \\
%%   @ NUM_ROUNDS_PER_DAY        @  &   50    &     $[2,\infty)$ (parell)  \\         
%%   @ BOARD_ROWS                @  &   15     &   $[12,25]$           \\
%%   @ BOARD_COLS                @  &   30     &   $[12,50]$                    \\

%%   @ NUM_INI_BUILDERS           @  &    4     &   $[1,6]$             \\
%%   @ NUM_INI_WARRIORS           @  &    2     &   $[1,4]$             \\
%%   @ NUM_INI_MONEY              @  &   10     &   $[0,10]$            \\
%%   @ NUM_INI_FOOD               @  &    5     &   $[0,10]$                    \\
%%   @ NUM_INI_GUNS               @  &    4     &   $[0,5]$                    \\
%%   @ NUM_INI_BAZOOKAS           @  &    2     &   $[0,4]$                    \\

%%   @ BUILDER_INI_LIFE           @  &    60   &    $[1,\infty)$                   \\ 
%%   @ WARRIOR_INI_LIFE           @  &    100   &   $[1,\infty)$                    \\
%%   @ MONEY_POINTS               @  &    5    &   $[1,\infty)$                    \\
%%   @ KILL_BUILDER_POINTS        @  &    100    &   $[1,\infty)$                    \\
%%   @ KILL_WARRIOR_POINTS        @  &    250   &   $[1,\infty)$                       \\
%%   @ FOOD_INCR_LIFE             @  &    20    &   $[1,\infty)$                   \\

%%   @ LIFE_LOST_IN_ATTACK        @  &    20    &   $[1,\infty)$                  \\

%%   @ BUILDER_STRENGTH_ATTACK    @  &    1     &  $[1,\infty)$                     \\
%%   @ HAMMER_STRENGTH_ATTACK     @  &    10     &  $[1,\infty)$                     \\
%%   @ GUN_STRENGTH_ATTACK        @  &    100     &  $[1,\infty)$                     \\
%%   @ BAZOOKA_STRENGTH_ATTACK    @  &    1000    &  $[1,\infty)$                     \\

%%   @ BUILDER_STRENGTH_DEMOLISH  @  &    3     &  $[1,\infty)$                     \\
%%   @ HAMMER_STRENGTH_DEMOLISH   @  &    10    &  $[1,\infty)$                \\
%%   @ GUN_STRENGTH_DEMOLISH      @  &    10    &  $[1,\infty)$                     \\
%%   @ BAZOOKA_STRENGTH_DEMOLISH  @  &    30    &  $[1,\infty)$                    \\

%%   @ NUM_ROUNDS_REGEN_BUILDER   @  &    50    &  $[1,\infty)$                     \\
%%   @ NUM_ROUNDS_REGEN_WARRIOR   @  &    50    &  $[1,\infty)$                     \\
%%   @ NUM_ROUNDS_REGEN_FOOD      @  &    10     &  $[1,\infty)$                     \\
%%   @ NUM_ROUNDS_REGEN_MONEY     @  &    5     &  $[1,\infty)$                     \\
%%   @ NUM_ROUNDS_REGEN_WEAPON    @  &    40   &   $[1,\infty)$                    \\ 
  
  
%%   @ BARRICADE_RESISTANCE_STEP  @  &    40    &  $[1,\infty)$                     \\
%%   @ BARRICADE_MAX_RESISTANCE   @  &    320   &  $[1,\infty)$                     \\
%%   @ MAX_NUM_BARRICADES         @  &    3     &  $[1,\infty)$                     \\


%%   \hline
%%   \end{tabular}
%%   \caption{Paràmetres del joc. La seva explicació la trobareu a l'arxiu \texttt{Settings.hh}.}
%%   \label{fig:parameters}
%% \end{figure}





%% \subsection{Afegir el vostre jugador}

%% Per crear un jugador nou 
%% copieu \texttt{AINull.cc} (un jugador buit que proporcionem com a plantilla)
%% a un fitxer nou \texttt{AIElquesigui.cc}.
%% A continuació, editeu el fitxer nou i canvieu la línia

%% \begin{center}
%% @#define PLAYER_NAME Null@
%% \end{center}

%% a

%% \begin{center}
%% @#define PLAYER_NAME Elquesigui@
%% \end{center}

%% El nom que trieu pel vostre jugador ha de ser únic, no ofensiu
%% i tenir com a màxim 12 caràcters. Aquest nom es mostrarà al lloc web
%% i durant les partides.

%% \medskip
%% A continuació, podeu començar a implementar el mètode virtual @play()@,
%% heretat de la classe base @Player@. Aquest mètode, que serà
%% cridat a cada ronda, ha de determinar les ordres que s'enviaran a les vostres unitats.

%% Podeu utilitzar definicions de tipus, variables i mètodes a
%% la vostra classe de jugador, però el punt d’entrada del vostre codi serà sempre el
%% mètode @play()@.



%% Des de la vostra classe jugador també podeu cridar funcions que
%% trobareu especificades als arxius següents:

%% \begin{itemize}
%% \item \texttt{State.hh}: accedir a
%%   l'estat del joc.
%% \item \texttt{Action.hh}: donar ordres als vostres ciutadans.
%% \item \texttt{Structs.hh}: estructures de dades útils.
%% \item \texttt{Settings.hh}: accedir als paràmetres del joc.
%% \item \texttt{Player.hh}: mètode @me()@.
%%   \item \texttt{Random.hh}: generar nombre aleatoris.
%% \end{itemize}

%% També podeu examinar el codi del
%% jugador ``Demo'' a \texttt{AIDemo.cc} com a exemple de com usar
%% aquestes funcions.

%% Tingueu en compte que no heu d'editar el mètode @factory()@ de la
%% classe del vostre jugador, ni l'última línia que afegeix el vostre
%% jugador a la llista de jugadors disponibles.


%% \subsection{Jugar contra el jugador Dummy}

%% Per a provar la vostra estratègia contra el jugador Dummy, proporcionem
%% el seu arxiu objecte. D'aquesta manera, no teniu accés al seu codi font
%% però podreu afegir-lo com a jugador i competir contra ell en local.

%% \medskip

%% Com ja hem comentat, per afegir el jugador Dummy a la llista de
%% jugadors registrats, heu de copiar l'arxiu corresponent a la vostra
%% arquitectura cap a \texttt{AIDummy.o}. Per exemple:

%% \texttt{cp AIDummy.o.Linux64 AIDummy.o}

%%   Recordeu que els arxius objecte contenen
%% instruccions binàries per a una arquitectura concreta, pel que no podem proporcionar
%% un únic arxiu.

%% \medskip

%% Consell de pro: demaneu als vostres amics els seus arxius {\bf
%%   objecte} (mai codi font!!!) i afegiu-los al vostre
%% \texttt{Makefile}!

%% \subsection{Restriccions en enviar el vostre jugador}

%% Quan creieu que el vostre jugador és prou fort per entrar a la
%% competició, podeu enviar-lo al Jutge. Degut a que s'executarà en un entorn segur
%% per prevenir trampes, algunes restriccions s'apliquen al vostre
%% codi:

%% \begin{itemize}

  
%% \item Tot el vostre codi font ha d’estar en un sol fitxer (com AIElquesigui.cc).

%% \item No podeu utilitzar variables globals (en el seu lloc, utilitzeu atributs a la vostra
%% classe).

%% \item Només teniu permès utilitzar biblioteques estàndard com
%% \texttt{iostream}, \texttt{vector}, \texttt{map}, \texttt{set},
%% \texttt{queue}, \texttt{algoritme}, \texttt{cmath}, \dots En molts
%% casos, ni tan sols cal incloure la biblioteca corresponent.

%% \item No podeu obrir fitxers ni fer cap altra crida a sistema (threads, forks, ...)
  
%% \item El vostre temps de CPU i la memòria que utilitzeu seran limitats, mentre que no ho són
%%   al vostre entorn local quan executeu \texttt{./Game}. El temps límit és d'un segon
%%   per l'execució de tota la partida. Si exhauriu el temps límit (o si l'execució del vostre codi
%%   avorta), el vostre jugador es congelarà i no admetrà més instruccions.

%% \item El vostre programa no ha d'escriure a @cout@ ni llegir de @cin@. Podeu
%%   escriure informació de depuració a @cerr@, però {\bf heu d'eliminar} aquesta informació
%%   en el codi que envieu al Jutge. 

%% \item Qualsevol enviament al Jutge ha de ser un intent honest de jugar.
%%   Qualsevol intent de fer trampes de qualsevol manera serà durament penalitzat.

%% \item Un cop hagueu enviat un jugador al Jutge que hagi derrotat al
%%   Dummy, podeu fer més enviaments però haureu de canviar el nom
%%   del jugador. És a dir, un cop un jugador ha vençut al Dummy, el
%%   seu nom queda bloquejat i no es pot reutilitzar.
%% \end{itemize}


