\section{El Visor}

A continuació descrivim el visor de partides:

\begin{itemize}
  
\item A la part superior hi ha botons que permeten
  reproduir o pausar la partida, anar al començament o al final de la
  partida, activar o desactivar el mode d’animació o obtenir una
  finestra d'ajuda amb més maneres de controlar com es reprodueix la
  partida. També hi trobareu la ronda actual i un botó per tancar el
  visor.
  Una barra de desplaçament horitzontal mostra visualment en quin punt
  de la partida es troba la ronda actual.
  
\item A la columna de l'esquerra, apareix cada jugador amb el nom i
  color corresponents. A sota es mostra la puntuació actual, el
  nombre d'unitats vives i la força del jugador corresponent. A les
  partides jugades a Jutge.org, també es mostra el percentatge de
  temps de CPU que s’ha consumit fins ara (si està esgotat, s’indica
  amb un 'out'). A la part superior dreta apareixen els colors dels jugadors
  ordenats per puntuació.
  
\item Les cel·les no posseïdes per ningú són de color blanc. En cas contrari, tenen el color del jugador que les posseeix.

\item Les cel·les amb deixalles tenen color gris fosc.

\item Les unitats vives es representen amb un cercle del color corresponent. En cas d'estar en procés de conversió, a zombi, tenen forma quadrada.

\item Les unitats mortes es representen amb una creu.

\item Els zombis es representen amb un quadrat vermell i contorn exterior negre.

\item Les unitats de menjar es representen amb un cercle vermell i contorn exterior negre.

\end{itemize}

%%% Local Variables:
%%% mode: latex
%%% TeX-master: t
%%% End:
