\section{Consells}

\begin{itemize}

\item {\bf NO DONEU O DEMANEU EL VOSTRE CODI A NINGÚ}. Ni tan sols una
  versió antiga. Ni fins i tot al vostre millor amic. Ni tans sols
  d'estudiants d'anys anteriors. Utilitzem detectors de plagi per
  comparar els vostres programes, també contra enviaments de jocs
  d’anys anteriors. No obstant, podeu compartir arxius objecte.

  
  Qualsevol plagi implicarà {\bf una nota de 0 en l'assignatura} (no
  només del Joc) de tots els estudiants involucrats. Es podran també
  prendre mesures disciplinàries addicionals. Si els estudiants A i B
  es veuen implicats en un plagi, les mesures s'aplicaran als dos,
  independentment de qui va crear el codi original. No es farà cap
  excepció sota cap circumstància.


\item Abans de competir amb els companys, concentreu-vos en derrotar al Dummy.
  
\item Llegiu les capçaleres de les classes que aneu a utilitzar. No cal que mireu 
les parts privades o la implementació.

\item Comenceu amb estratègies simples, fàcils d'implementar i depurar, ja que
és exactament el que necessitareu al principi.

\item Definiu mètodes auxiliars senzills (però útils) i \emph{assegureu-vos
que funcionin correctament}.

\item Intenteu mantenir el vostre codi net. Això farà més fàcil canviar-lo i afegir
  noves estratègies.

\item Com sempre, compileu i proveu el vostre codi sovint. És \emph{molt}
  més fàcil rastrejar un error quan només heu canviat poques línies de codi.

\item Utilitzeu @cerr@ per produir informació de depuració i afegiu
  \texttt{assert}s per assegurar-vos que el vostre codi fa el que hauria de fer.

\item Quan depureu un jugador, elimineu els @cerr@s que tingueu en el codi
d’altres jugadors, per tal de veure només els missatges que desitgeu.

\item Podeu utilitzar comandes com el \texttt{grep} de Linux per tal de filtrar
  la sortida produïda per \texttt{Game}.

\item Activeu l'opció \texttt{DEBUG} al Makefile, que us permetrà
 obtenir traces útils quan el vostre programa avorta. També hi ha una opció
\texttt{PROFILE} que podeu utilitzar per optimitzar codi.

\item Si l'ús de @cerr@ no és suficient per depurar el vostre codi, apreneu
com utilitzar \texttt{valgrind}, \texttt{gdb} o qualsevol altra eina de depuració.


\item Podeu analitzar els arxius produïts per \texttt{Game}, que descriuen com
  evoluciona el tauler a cada ronda.
  
\item Conserveu una còpia de les versions antigues del vostre jugador. Feu-lo lluitar
contra les seves versions anteriors per quantificar les millores.


\item Assegureu-vos que el vostre programa sigui prou ràpid. El temps de CPU que
es permet utilitzar és bastant curt.

\item Intenteu esbrinar les estratègies dels altres jugadors observant
  diverses partides. D’aquesta manera, podeu intentar reaccionar als
  seus moviments, o fins i tot imiteu-los o milloreu-los amb el vostre
  propi codi.

\item No espereu fins al darrer minut per enviar el jugador. Quan hi
  ha molts enviaments al mateix temps, el servidor triga més en
  executar les partides i podria ser ja massa tard!

\item Podeu enviar noves versions del vostre programa en qualsevol moment.
  
\item Recordeu: mantingueu el codi senzill, compileu-lo sovint i proveu-lo sovint, o
us en penedireu.

\end{itemize}



