\Problem{El campanar de la Torrassa}


\Statement


\FigureR{width=5cm}{campanar} 
\emph{El Campanar de la Torrassa} (see picture) is the bell tower of a church
located in the district of La Torrassa in the city of L'Hospitalet de
Llobregat. This bell tower is well known for continuously annoying its
neighbors with the sound of its bells.

Day or night, the bells sound every quarter in the traditional way: Suppose it
is seven in the afternoon (19:00). In this case, at 19:00 the treble bells
sound 4 times and the bass bells sound 7 times. Then, at 19:15 the treble
bells sound once. Latter, at 19:30 the treble bells sound twice. Finally, at
19:45 the treble bells sound three times.  Additionally, each day at noon the
bass bells sound 100 times rather than 12 to signal the \emph{Angelus}. The
bells always finish playing within the first minute (that is, at 19:01, 19:16,
19:31, 19:46, etc.).

\medskip

(This is a simplification, because on Saturdays, Sunday and important dates
extra bells are played to signal special masses, not to count the 15th of
August, when they play all day long.)

\medskip

A non governmental organization that fights acoustic pollution wants to count
the number of times that the bells of the bell tower sound in a large amount of
time. Specifically, they need a program, that given a starting time and a
length, computes the number of times the bells sound in this period of time.

\Input

Input is made of several test cases, each one on a line. Each test case
consists of three integer numbers:
$h$ and $m$ encode the starting
time ($h$:$m$) and satisfy $0\le h\le 23$ and $0\le m\le 59$;
$\ell$ encodes the length, in
minutes, of the time we want to measure and satisfies $0\le\ell\le2^{28}$.

\Output

For each test case, output must include a line with an integer denoting the
number of times that the bells sound starting at $h$:$m$ for a length of $\ell$
minutes.Sample

\Observation

This problem is from the 2nd UPC Programming Competition.
