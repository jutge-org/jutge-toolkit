\Problem{El campanar de la Torrassa}


\Statement


\FigureR{width=5cm}{campanar} 
\emph{El Campanar de la Torrassa} és un campanar que es troba al barri de
Collblanc-La Torrassa de L'Hospitalet de Llobregat. Aquest campanar és ben
conegut per molestar continuament a tots els seus veïns amb les seves
campanades.
Dia i nit, les campanes sonen cada quart d'hora de la forma
tradicional: Suposeu que són les set de la tarda. En aquest
cas, a les 19:00 les campanes agudes toquen quatre cops i les campanes
greus toquen 7 cops. Després, a un quart de vuit (19:15),
les campanes agudes toquen un cop. Més tard, a dos quarts de vuit
(19:30) les campanes agudes toquen dos cops. Finalment,
a tres quart de vuit (19:45), les campanes agudes toquen tres cops.
A més a més, cada dia al migdia les campanes greus toquen 100 cops enlloc
de 12 per tal de senyalar l'\emph{Àngelus}. Les campanes sempre
acaben de tocar en menys d'un minut.

\medskip

(De fet, aquesta situació és una simplificació, perquè en diumenges
i festes assenyalades, hi ha campanades extra per les misses 
especials, sense parlar del quinze d'agost, quan les campanes toquen durant
tot el dia.)

\medskip

Una ONG que lluita contra la pol·lució acústica a les ciutats
vol comptar el nombre de cops que les campanes d'aquest campanar
sonen en un gran interval de temps. Concretament, necessiten un
programa que, donada una hora d'inici i una llargada de temps,
calculi el nombre de campanades que sonen en aquest periode de temps.


\Input

L'entrada conté diferents jocs de proves, un per línia. Cada joc de proves
consisteix en tres enters: $h$ i $m$ codefiquen l'hora d'inici ($h$:$m$) i
satisfàn $0\le h\le 23$ i $0\le m\le 59$; $t$ codefica la llargada de temps
(en minuts) del temps que es vol mesurar i satisfà $0\le t\le2^{28}$. 

\Output

Per a cada joc de proves, cal escriure en una línia un enter corresponent
al nombre de cops que les campanades toquen començant a l'hora $h$:$m$ 
i durant un periode de $t$
minuts.


\Sample

\Observation

Aquest problema va aparèixer a les semifinals del 2n Concurs de Programació de
la UPC.
