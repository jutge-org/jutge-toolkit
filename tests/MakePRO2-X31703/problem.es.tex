\Problem{Práctica de PRO2 - Primavera 2023 - entrega final provisional}

\Statement


Este problema permite hacer entregas de prueba de la práctica completa. 
Tened en cuenta que:

\begin{itemize}
\item
no es el canal para hacer la entrega definitiva de la práctica

\item
el problema del Jutge de la entrega definitiva {\bf puede contener
juegos de prueba o condiciones diferentes} de los que aparecen en este
problema

\item
las entregas realizadas en este problema no serán tenidas en cuenta
para la nota de la práctica

\item
en esta entrega no pedimos carpetas generadas mediante {\tt doxygen},
pero los ficheros de las clases pueden llevar todos los comentarios
{\tt doxygen} que queráis; de hecho recomendamos que incluyáis las
especificaciones de todas las operaciones y que sean lo más
definitivas posible

\end{itemize}

\Input

Una secuencia de instrucciones y datos que siguen el formato del enunciado de
la práctica y del juego de pruebas público.

\Output

Una secuencia de resultados que siguen el formato del enunciado de la
práctica y del juego de pruebas público.

\Observation

El Jutge prueba vuestras entregas mediante 4 juegos de pruebas:


\begin{itemize}
\item
  sample: el juego de pruebas público
\item
  privat1: combinación de todos los juegos de pruebas de la entrega intermedia
\item
  privat2: un poco de todo, salvo eficiencia
\item
  privat3:  eficiencia 

\end{itemize}


En un fichero llamado {\tt practica.tar} tenéis que entregar
\begin{itemize}
\item
  Los ficheros {.hh} y {.cc}  de las clases y el programa principal
\item
  El fichero Makefile, que usaremos para generar y probar el ejecutable
\end{itemize}
Tened en cuenta  las siguientes restricciones:
\begin{itemize}
\item
  El fichero que contiene el programa principal se ha de llamar {\tt program.cc}
\item
  El Makefile ha de generar un ejecutable llamado {\tt program.exe} 
\item
  Es importante que uséis las opciones de compilación definidas en el menú Documentation $\rightarrow$ Compilers $\rightarrow$ PRO2 de {\tt www.jutge.org} para funcionar en igualdad de condiciones con el Jutge
  %Altrament us arrisqueu a patir dos tipus de problemes: excés de temps durant la compilació ({\tt compilation time exceeded}) o excés de temps durant l'execució.
\item
  No usar la opción {\tt -D\_GLIBCXX\_DEBUG} o usarla de forma incorrecta
  podrá ser penalizado
\end{itemize}

Producid el fichero {\tt practica.tar} con la instrucción Linux
\begin{verbatim}
tar -cvf practica.tar fitxer1 fitxer2 fitxer3 ...
\end{verbatim}
desde el directorio/carpeta donde tengáis los ficheros que vais a
entregar.  Incluid en vuestro Makefile una regla con esta instrucción,
de forma que el {\tt .tar} se pueda generar ejecutando {\tt make
practica.tar}. Con eso reduciréis el riesgo de error en sucesivas
entregas.  El Jutge no acepta {\tt .tar} donde los ficheros estén
dentro de carpetas. Recomendamos usar {\tt GNU tar} para reducir el
riesgo de que el fichero {\tt practica.tar} sea incompatible con el
Jutge. 


\Sample

