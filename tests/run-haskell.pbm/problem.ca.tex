
\Problem{Haskell --- Funcions amb nombres}

\UseHaskell

\Statement

En aquest problema heu d’implementar una sèrie de funcions en Haskell.
No cal que pregunteu si podeu fer servir funcions auxiliars, és evident que sí.



\begin{enumerate}

\item Feu una funció @absValue :: Int -> Int@ que,
donat un enter, retorni el seu valor absolut.

\item Feu una funció @power :: Int -> Int -> Int@ que,
donats un enter $x$ i un natural $p$, retorni $x$ elevat a $p$, és a dir, $x^p$.


\item Feu una funció @isPrime :: Int -> Bool@ que,
donat un natural, indiqui si aquest és primer o no.


\item Feu una funció @slowFib :: Int -> Int@ que
retorni l'$n$-èsim element de la sèrie de Fibonacci
tot utilitzant l'algorisme recursiu que la defineix
($f(0)=0$, $f(1)=1$, $f(n)=f(n-1)+f(n-2)$ per $n\ge 2$).


\item Feu una funció @quickFib :: Int -> Int@ que
retorni l'$n$-èsim element de la sèrie de Fibonacci
tot utilitzant un algorisme més eficient.


\end{enumerate}

\Scoring

Cada funció puntua 20 punts.

\SampleOneCol
